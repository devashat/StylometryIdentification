\documentclass[12pt]{article}

\usepackage{fullpage}
\usepackage[T1]{fontenc}
\usepackage{mathptmx}

\title{\textbf{\underline{Assignment 7 Design Document}}}
\author{Devasha Trivedi}
\date{\today}

\begin{document}\maketitle
This assignment has us implement a program that identifies the most likely authors of a given text using stylometry.

\section{Files to Submit}\label{ss:deliverables}
\begin{enumerate}
    \item bf.h: This header file was provided to us and contains the interface for the Bloom ADT.
    \item bf.c: This file contains my implementation of the Bloom filter ADT.
    \item bv.h: This header file was provided to us and contains the interface for the bit vector ADT.
    \item bv.c: This file contains my implementation of the bit vector ADT.
    \item ht.h: This header file was provided to us and contains the interface for the hash table ADT.
    \item ht.c: This file contains my implementation of the hash table ADT.
    \item identify.c: This file contains my implementation fo the author identificaiton program.
    \item metric.h: This header file was provided to us and contains definitions for distance metrics.
    \item node.h: This header file was provided to us and contains the interface for the node ADT.
    \item node.c: This file contains my implementation of the node ADT.
    \item parser.h: This header file was provided to us and contains the interface for the parsing module.
    \item parser.c: This file contains the parsing module implementation.
    \item pq.h: This header file was provided to us and contains the interface for the priority queue ADT.
    \item pq.c: This file contains my implementation of the priority queue ADT.
    \item salts.h: This header file was provided to us and contains the definitions for the salts to be used in bf.c, as well as in ht.c
    \item speck.h: This header file was provided to us and contains the interface for the SPECK cipher.
    \item speck.c: This file contains the SPECK cipher implementation.
    \item text.h: This header file was provided to us and contains the interface for the text ADT.
    \item text.c: This file contains my implementation of the text ADT.
    \item Makefile: This file helps compile and run my program.
    \item README.md: This file contains a summary of the program, as well as how to interact with the Makefile.
    \item DESIGN.pdf: This file contains the pseudocode of my implementations.
    \item WRITEUP.pdf: This file discusses my program's behavior. It also serves as a reflection of how I coded the program and how it works.
\end{enumerate}

\section{Pseudocode}\label{ss:pseudocode}
\subsection{\textbf{bf.c}}
\begin{verbatim}
bf_create
    use salts to initialize bf
bf_delete 
    bv_delete(bf)
    free(bf)
bf_size
    bv_length
bf_insert
    set bits for:
    primary
    secondary
    tertiary
\end{verbatim}

\subsection{\textbf{bv.c}}
\begin{verbatim}
bv_create
    set vector
    set length
    for i <= length:
      clr_bit()
bv_delete
    free(bv)
bv_length
    return length(bv)
refer to code comments for:
set_bit
clr_bit
get_bit
\end{verbatim}

\subsection{\textbf{ht.c}}
\begin{verbatim}
use salts to initialize

ht_delete
node_delete
free(ht)

ht_size
return length(ht)

ht_lookup
while node != NULL:
  index++
  index %= ht_size
  node = ht_slots
return node

ht_insert
same while loop from ht_lookup
if node = NULL
  node = node_create(word)
  node_count++
  ht_slots = node
return node


HTI:

table = ht
slot = 0

hti_delete
free(hti)

ht_iter
node = table->slots[slot]
slot++
return node
\end{verbatim}

\subsection{\textbf{node.c}}
\begin{verbatim}
create a node
    -use malloc
    -set word
   -set a counter variable
delete a node
    -use free
    -set node to NULL
\end{verbatim}

\subsection{\textbf{pq.c}}
\begin{verbatim}
create pq
    -set capacity
    -set counter variable
    -set nodes
delete pq
    -use free
    -set everything to NULL
is pq empty
    empty = true
    if size(pq) > 0:
        empty = false
    return empty
is pq full
    empty = true
    if size(pq) < pq(capacity):
        empty = false
    return empty
enqueue
    build heap from asgn3
    heap goes up
dequeue
    build heap from asgn3
    heap goes down
\end{verbatim}

\subsection{\textbf{text.c}}
\begin{verbatim}
int ht_size, bf_size

create text->ht and text->bf
word_count = 0

use a while loop and a bool
    -insert ht and bf
    -return text

delete text
    ht_delete
    bf_delete

total_count
    use hti and for loop
    for i < ht_size:
      Node *n = ht_iter(hti);
      if n != NULL:
        total_count += n->count;
    return total_count;

text_distance
    calculate all metrics
    if statements to set them

text_frequency
    if text has a word:
      Node *slot = ht_lookup
        if slot != NULL
          freq = slot_size
    return freq
\end{verbatim}

\subsection{\textbf{identify.c}}
\begin{verbatim}
main() is gonna have getopt() loop
getopt() loop has command-line parsing

create a noise text & anonymous text
create a pq

for i < author_count:
    get data required

    if author = NULL:
        print(File cant be opened)
    else:
        enqueue author and distance

    print matches, metric names
    dequeue author and distance 
\end{verbatim}

\section(Credit)\label(ss:credit)
\begin{itemize}
    \item I referred to the bv8.h file in the code comments repo for the set, clr, and get bit functions in bv.c
    \item I went to Eugene's section to understand the assignment and some pseudocode.
    \item I referred to various chapters from the textbook to refresh my memory/assist my coding.
    \item I took node.c and pq.c from my asgn6 code.
    \item I used the heapsort logic from asgn3 for pq.c
\end{itemize}

\end{document}
